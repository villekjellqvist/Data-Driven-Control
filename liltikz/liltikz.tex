\documentclass[tikz]{standalone}

\usepackage{xfrac}
\begin{document}

\usetikzlibrary{arrows.meta}
\usetikzlibrary{positioning} 
\tikzset{>={Latex[width=2mm,length=2mm]}}

\tikzstyle{block} = [draw, fill=white, rectangle, 
    minimum height=3em, minimum width=6em,]
\tikzstyle{sum} = [draw, fill=white, circle]
\tikzstyle{input} = [coordinate]
\tikzstyle{output} = [coordinate]
\tikzstyle{pinstyle} = [pin edge={to-,thin,black}]

\begin{tikzpicture}[auto, node distance=2.5cm]
    \node[input, name=input] {};
    \node[block, right of=input] (controller) {$Ru = Tu_c - Sy$};
    \node[sum, right of=controller] (dsum) {$\Sigma$};
    \node[block, right of=dsum] (Ho)
        {$H_o = \frac{z+1.2}{z^2 -z + \sfrac{1}{4}}$};
    \node[input, above of=dsum, node distance=1.2cm] (v) {};
    \node[output, right of=Ho] (output) {};
    \node[coordinate, below of=dsum, node distance=1.5cm] (mid) {};
    \draw [->] (controller) -- node[name=u] {$u$} (dsum);
    \draw [->] (dsum) -- (Ho);
    \draw [->] (v) -- node[name=vv] {$v$} (dsum);
    \draw [->] (Ho) --node[name=y] {$y$} (output);
    \draw [-] (y) |- (mid);
    \draw [->] (mid) -| (controller);
    \draw [->] (input) -- node[name=uc] {$u_c$} (controller);



\end{tikzpicture}

\end{document}